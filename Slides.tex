\documentclass{beamer}

\usepackage{lmodern}
\usepackage[spanish]{babel}
\usepackage[utf8]{inputenc}
\usepackage[T1]{fontenc}
\usepackage{hyperref}
\usepackage{bm}

\newcommand{\hrefCustom}[2]{\href{#1}{\color{blue} #2}}

\setlength{\parskip}{1em}
\usetheme{Singapore}

\definecolor{title_color}{HTML}{3F51B5}
\setbeamercolor{titlelike}{fg=title_color}

\definecolor{structure_color}{HTML}{F44336}
\setbeamercolor{structure}{fg=structure_color}

\definecolor{author_color}{HTML}{4CAF50}
\setbeamercolor{author}{fg=author_color}

\definecolor{institute_color}{HTML}{FFC107}
\setbeamercolor{institute}{fg=institute_color} 

\title{Introduccion a la Ingeniería Inversa}
\subtitle{Análisis de malware, cracking de software...}
\author{Carlos Ledesma Peña
\and
Fernando Díaz Urbano}
\institute{Grupo de Desarrollo de \raisebox{-0.5ex}{\includegraphics[scale=0.22]{Logo_Google_2013_Official.png}}, Málaga}
\date{}

%\AtBeginSection[]
%{
%	\begin{frame}[shrink=37]
%		\frametitle{Índice}
%		\vspace{2em}
%		\tableofcontents[currentsection]
%	\end{frame}
%}

\AtBeginSubsection[]
{
	\begin{frame}[shrink=37]
		\frametitle{Índice}
		\vspace{2em}
		\tableofcontents[currentsection,currentsubsection]
	\end{frame}
}

% To include images from 'images' directory
\graphicspath{ {./images/} }

\begin{document}

\begin{frame}
\titlepage
\end{frame}

\section{Introducción}

\subsection{¿Qué es la ingenieria inversa?}

\begin{frame}{Antecedentes}

\begin{columns}

\begin{column}{.50\textwidth}
\begin{center}
\includegraphics[scale=0.15]{hires-wehr3.jpg}
\end{center}
\end{column}

\begin{column}{.50\textwidth}
\begin{figure}
\vspace{-1ex}\hspace{-2.5ex}\includegraphics[scale=0.11]{K-13.jpg}
\end{figure}

\vspace{2ex}
\textbf{Izquierda}: Máquina Enigma versión militar alemana

\vspace{1ex}

\textbf{Arriba}: Misil K-13 soviético aire-aire 

\end{column}

\end{columns}

\end{frame}

\begin{frame}{Aplicaciones en informática}
Este concepto se puede aplicar a diversas aplicaciones, se viene haciendo desde tiempos inmemoriales.
\begin{itemize}
	\item \textbf{Dispositivos electronicos} 
	\item \textbf{Programas informáticos}
	\item \textbf{Artilugios de uso cotidiano}
\end{itemize}
\end{frame}

\begin{frame}{¿Qué herramientas necesitamos?}
\begin{itemize}
\item \textbf{Conocer la API de Windows(Si el reversing se hace en Windows)}
\item \textbf{Debuggers}
\item \textbf{Conocimientos en Assembly}
\item \textbf{¡Un cerebro!}
\end{itemize}

\end{frame}

%Sugiero basar la presentación en explicar por encima cada uno de estos aspectos, además de explicar conceptos como los de análisis estático y dinámico y las demostraciones
\begin{frame}{¿Qué debemos conocer?}
\begin{itemize}
\item \textbf{El entorno} \\ \hspace{4ex}Sistema operativo, proceso, ejecutable, librería...
\item \textbf{Los lenguajes} \\ \hspace{4ex}Ensamblador (x86), bytecode (CIL), scripting (Perl)...
\item \textbf{Los patrones típicos} \\ \hspace{4ex}Reconocer un bucle while, una inyección en un proceso...
\item \textbf{Las herramientas} \\ \hspace{4ex}Desensamblador, depurador, monitor de recursos...
\end{itemize}

\end{frame}



\subsection{Conceptos básicos}

\begin{frame}{Análisis estático VS Dinámico}
Estatico:
\begin{itemize}
	\item \textbf{Permite examinar todos los valores de variables y caminos}
	\item \textbf{Encontrar errores que no se manifestarán hasta pasado mucho tiempo}
\end{itemize}
Dinámico
\begin{itemize}
	\item \textbf{Permite revelar defectos o vulnerabilidades difíciles de encontrar estáticamete}
	\item \textbf{Observar el comportamiento deun ejecutable encriptado}
\end{itemize}
Necesitamos de ambos, para poder llevar a cabo un análisis completo.


\end{frame}

\begin{frame}{Opcodes}

\begin{center}
\includegraphics[scale=0.3]{0cPOx.png}
\end{center}

\end{frame}

\begin{frame}{Bytecode}

\begin{center}
\includegraphics[scale=0.3]{java-program-execution2.png}
\end{center}

\end{frame}

\begin{frame}{WinDBG}

Esta herramienta desarrollada por Microsoft, nos permite:
\begin{itemize}
\item \textbf{Debuggear a nivel de Kernel(Ring 0)}
\item \textbf{Debuggear el propio Sistema Operativa}
\item \textbf{No tiene GUI, funciona a traves de una consola}
\item \textbf{Es complicado para principiantes, tedioso}
\end{itemize}
\begin{center}
\includegraphics[scale=0.21]{windbg.png}
\end{center}

\end{frame}

\section{Evolución en el tiempo de la tecnología SDR}

\begin{frame}{Ollydbg}

\begin{itemize}
	\item \textbf{Posee GUI}
	\item \textbf{User-level debugging(Ring 3)} 
	\item \textbf{Más sencillo para principiantes}
\end{itemize}
\includegraphics[scale=0.25]{OllyDbg2.png}
\end{frame}

\subsection{Comienzos}

\begin{frame}{Anillos de protección}
 \begin{center}
\includegraphics[scale=0.35]{Priv_rings.png}
\end{center}
\textbf{HyperVisor}
\begin{itemize}
\item \textbf{AMD-V}: Pacifica
\item \textbf{Intel VT-x}: Vanderpool
\end{itemize}
\end{frame}

\subsection{Seguridad}

\begin{frame}{Seguridad}

\begin{itemize}
	\item \textbf{2005:} Versión mejorada de \emph{BlueSniper}.
	\item \textbf{2007:} Ataque sobre teclados inalámbricos.
	\item Ataque sobre el pasaporte europeo.
	\item \textbf{2008:} Michael Ossmann repasa en Black Hat sobre el estado de la seguridad de las radiocomunicaciones, y advierte que SDR accesible es peligroso.
	\item Ataque sobre el sistema de tarjetas del metro de Boston y de pago remoto en peajes.
	\item \textbf{2009:} Ataque práctico sobre \emph{GSM}.
	\item \textbf{2010:} Lectura de \emph{RFID} a larga distancia.
	\item \textbf{2011:} Ataque sobre \emph{GRPS/EDGE} y \emph{UMTS/HSPA}.
\end{itemize}

\end{frame}

\subsection{Estado actual}

\begin{frame}{Estado actual}

\begin{itemize}
	\item En 2010, Eric Fry se da cuenta de algo extraño al realizar ingeniería inversa a un \emph{driver} de un dispositivo \emph{USB} para recepción \emph{FM} y \emph{DAB+}. Lo que viaja del dispositivo al PC no es audio, sino muestras de la señal en una etapa intermedia entre la señal de radiofrecuencia y el audio.
	\item En 2012 nace el proyecto \emph{rtl-sdr}, que proporciona una interfaz para usar estos dispositivos como SDR's (sólo recepción, pero muy asequibles).
	\item Interés en integrar SDR en la comunidad de pentesting, con nuevas herramientas que permiten inyectar paquetes de diversos protocolos al vuelo.
\end{itemize}

\end{frame}

\section{Caso práctico}

\subsection{GNU Radio}

\begin{frame}{¿Qué es GNU Radio?}

\emph{GNU Radio} es un entorno de desarrollo \emph{open source} multiplataforma de procesamiento de señales en general, si bien está especializado en SDR, pero no limitado a ello.

\begin{itemize}
	\item Ofrece una interfaz gráfica, \emph{GRC}, además de las interfaces para \emph{Python} y \emph{C++}. La de \emph{Python} es una envoltura de la de \emph{C++}, y la gráfica una envoltura de la de \emph{Python}.
	\item La interfaz gráfica sirve para crear diagramas de flujo, con conexiones entre bloques que representan funciones de procesamiento de señales.
	\item Los bloques pueden ser de entrada o salida, para interactuar con el exterior (parte hardware de SDR, tarjeta de audio, disco duro...), o de entrada y salida, implementando funciones en sí.
\end{itemize}

\end{frame}

\begin{frame}{GRC}

\begin{center}
\includegraphics[scale=0.3]{grc.png}
\end{center}

\end{frame}

\subsection{Pager POCSAG}

\begin{frame}{¿Qué es un pager POCSAG?}

\begin{center}
\includegraphics[scale=0.08]{pagerfront.jpg}
\end{center}

\end{frame}

\begin{frame}{Al ataque}

\begin{itemize}
	\item Tras escanear en el rango de frecuencias que menciona el \emph{pager} en su parte de atrás, y los avisos que alcanzo a capturar se emiten en la misma frecuencia.
	\item Construyo el diagrama de flujo (no sin mucho esfuerzo) para decodificar con arreglo al estándar \emph{POCSAG} y efectivamente, se ajusta al estándar. Cada disposivo tiene un ID, y suena cuando se emite el suyo.
	\item Modifico el diagrama de flujo y creo un bloque personalizado para \emph{GRC} para imprimir en consola los ID's según se capturan los avisos.
\end{itemize}

\end{frame}

\begin{frame}{Dificultades encontradas}

\begin{itemize}
	\item Dominio completamente nuevo para mí, y falta de base sólida a la hora de resolver los problemas (días de diagnóstico por problema).
	\item \emph{GRC} no está hecho para aprender a base de prueba y error desde el principio, no es fácil saber qué está fallando ni por qué (curva de aprendizaje elevada).
	\item Limitaciones del hardware, mi portátil usa \emph{USB 2.0}, lo que limita el ancho de banda capturable de una vez, además de no tener potencia de procesamiento suficiente y descartar muestras si se usaban varias operaciones simultáneamente.
\end{itemize}


\end{frame}

\section{Para terminar}

\subsection{Conclusiones}

\begin{frame}{Conclusiones}

\begin{itemize}
	\item Desde el punto de vista económico y humano, es necesario invertir en la seguridad de los sistemas informáticos. No sólo se protege de las malas intenciones, sino de las buenas intenciones equivocadas.
	\item La ''seguridad'' por oscuridad no es seguridad, si un sistema necesita que su forma de funcionar no sea pública para ser seguro, no es seguro igualmente. \emph{Sólo la clave debe ser desconocida para el resto}.
	\item No se le ha prestado suficiente atención a la seguridad de las radiocomunicaciones en el pasado, y ahora se dispone de herramientas basadas en SDR que facilitan aprovecharse de sistemas vulnerables. Hay que prestarle atención desde ya.
\end{itemize}

\end{frame}

\subsection{Utilidades}

\begin{frame}{Utilidades}

\begin{itemize}
\setlength{\itemsep}{12pt}
	\item Aprender conocimientos básicos de radio (inquietud personal).
	\item Incorporar una nueva herramienta de trabajo (es posible que se incorpore en labores de pentesting).
	\item Servir de guía de inicio rápido a SDR a los investigadores del departamento.
	\item Obtener el título de Graduado en Ingeniería Informática.
\end{itemize}

\end{frame}

\begin{frame}

\begin{center}
\textbf{¡Gracias por vuestra atención!}
\end{center}

\end{frame}

\end{document}
