\documentclass{beamer}

\usepackage{lmodern}
\usepackage[spanish]{babel}
\usepackage[utf8]{inputenc}
\usepackage[T1]{fontenc}
\usepackage{hyperref}
\usepackage{bm}
\usepackage{listings}

\newcommand{\hrefCustom}[2]{\href{#1}{\color{blue} #2}}

\setlength{\parskip}{1em}
\usetheme{Singapore}

\definecolor{title_color}{HTML}{3F51B5}
\setbeamercolor{titlelike}{fg=title_color}

\definecolor{structure_color}{HTML}{F44336}
\setbeamercolor{structure}{fg=structure_color}

\definecolor{author_color}{HTML}{4CAF50}
\setbeamercolor{author}{fg=author_color}

\definecolor{institute_color}{HTML}{FFC107}
\setbeamercolor{institute}{fg=structure_color} 

\title{Introduccion a la Ingeniería Inversa}
\subtitle{Análisis de malware, cracking de software...}
\author{Carlos Ledesma Peña
\and
Fernando Díaz Urbano}
\institute{Grupo de Desarrolladores de \raisebox{-0.45ex}{\includegraphics[scale=0.22]{Logo_Google_2013_Official.png}}, Málaga}
\date{}

%\AtBeginSection[]
%{
%	\begin{frame}[shrink=37]
%		\frametitle{Índice}
%		\vspace{2em}
%		\tableofcontents[currentsection]
%	\end{frame}
%}

%\AtBeginSubsection[]
%{
%	\begin{frame}[shrink=37]
%		\frametitle{Índice}
%		\vspace{2em}
%		\tableofcontents[currentsection,currentsubsection]
%	\end{frame}
%}

% To include images from 'images' directory
\graphicspath{ {./images/} }

\begin{document}

\begin{frame}
\titlepage
\end{frame}

\section{Introducción}

\begin{frame}{Antecedentes}

\begin{columns}

\begin{column}{.50\textwidth}
\begin{center}
\includegraphics[scale=0.1]{K-13.jpg}
\vspace*{2ex}
\\\textbf{Misil K-13 soviético aire-aire}
\end{center}
\end{column}

\begin{column}{.50\textwidth}
\begin{center}
\includegraphics[scale=0.82]{stripped.png}
\vspace*{2ex}
\\\textbf{CDP MPC 1600-1}
\end{center}
\end{column}

\end{columns}

\end{frame}

\begin{frame}{Aplicaciones típicas en informática}
\begin{itemize}
	\item \textbf{Mantenimiento de programas sin código fuente}\\ \hspace{4ex}Interacción de componentes, creación de documentación...
	\item \textbf{Análisis de malware}\\ \hspace{4ex}Conocer objetivos y procedimientos de un malware
	\item \textbf{Búsqueda de vulnerabilidades}\\ \hspace{4ex}En programas de código cerrado
	\item \textbf{Modding y cracking}\\ \hspace{4ex}Añadir o remover funcionalidades
\end{itemize}
\begin{center}... Entre otras.\end{center}
\end{frame}

\begin{frame}{Aplicaciones típicas en informática}
\begin{itemize}
	\item \textbf{Mantenimiento de programas sin código fuente}\\ \hspace{4ex}Interacción de componentes, creación de documentación...
	\item \textbf{Análisis de malware}\\ \hspace{4ex}Conocer objetivos y procedimientos de un malware
	\item \textbf{Búsqueda de vulnerabilidades}\\ \hspace{4ex}En programas de código cerrado
	\item \textbf{\textcolor{structure_color}{Modding y cracking}}\\ \hspace{4ex}Añadir o remover funcionalidades
\end{itemize}
\begin{center}... Entre otras.\end{center}
\end{frame}

%Sugiero basar la presentación en explicar por encima cada uno de estos aspectos, además de explicar conceptos como los de análisis estático y dinámico y las demostraciones
\begin{frame}{¿Qué debemos conocer?}
\begin{itemize}
\item \textbf{El entorno} \\ \hspace{4ex}Proceso, ejecutable, thread, librería...
\item \textbf{Las herramientas} \\ \hspace{4ex}Desensamblador, depurador, monitor de recursos...
\item \textbf{Los lenguajes} \\ \hspace{4ex}Ensamblador (x86), bytecode (JavaVM), scripting...
\item \textbf{Los patrones típicos} \\ \hspace{4ex}Reconocer un bucle for, uso típico de la API Win32...
\end{itemize}

\end{frame}

\begin{frame}{Tipos de análisis}
\vspace{-3ex}
\textbf{Estático}
\begin{itemize}
	\item Permite examinar todos los caminos y valores de variables
	\item Examen directo de ramas de ejecución poco frecuentes
	\item No efectivo contra código automodificable o muy ofuscado
\end{itemize}
\textbf{Dinámico}
\begin{itemize}
	\item Sólo se ejecuta un camino cada vez, dependiente del entorno
	\item Más fácil entender lógica complicada viendo la ejecución
	\item Efectivo contra código automodificable u ofuscado
\end{itemize}

\end{frame}

\section{Entorno}

\begin{frame}[plain]
\begin{center}
\includegraphics[scale=0.4]{pe_sections.png}
\end{center}
\end{frame}

\begin{frame}{Ejecutables, procesos y threads}

\begin{columns}

\begin{column}{.50\textwidth}
\begin{center}
\includegraphics[scale=0.35]{threads.png}
\end{center}
\end{column}

\begin{column}{.50\textwidth}
\begin{itemize}
\item \textbf{Ejecutable}\\ \hspace{4ex}Contenedor de código,\\\hspace{4ex}''plantilla'' de procesos
\item \textbf{Proceso}\\ \hspace{4ex}Rango de memoria,\\\hspace{4ex}contenedor de threads
\item \textbf{Thread}\\ \hspace{4ex}Contexto de ejecución,\\\hspace{4ex}corren en paralelo
\end{itemize}
\end{column}

\end{columns}

\end{frame}

\begin{frame}{Anillos de protección}
 \begin{center}
\includegraphics[scale=0.35]{Priv_rings.png}
\end{center}
\end{frame}

\begin{frame}{Win32 API}
\vspace{-4ex}
\begin{center}
Casi desde Windows 95 hasta Windows 10...
\end{center}
\vspace{-2ex}
\textbf{kernel32.dll}
\begin{itemize}
	\item Procesos: \textit{CreateProcess, CreateThread, LoadLibrary...}
	\item Archivos: \textit{CreateFile, CopyFile, GetFilesize...}
	\item Memoria: \textit{VirtualAlloc, HeapAlloc, MapViewOfFile...}
\end{itemize}
\textbf{user32.dll}
\begin{itemize}
	\item Iniciar GUI's: \textit{CreateWindowEx, MessageBox, LoadCursor...}
	\item Gestionar GUI's: \textit{GetMessage, PeekMessage, PostMessage...}
\end{itemize}

\end{frame}

\section{Herramientas}

\begin{frame}{WinDBG}

\begin{itemize}
\item Depurar ("debuggear") a nivel de kernel (ring 0)
\item Debuggear el propio sistema operativo, drivers...
\item Curva de aprendizaje empinada, tedioso...
\item No tiene GUI, funciona a través de una consola
\end{itemize}
\begin{center}
\includegraphics[scale=0.21]{windbg.png}
\end{center}

\end{frame}

\begin{frame}{OllyDBG}

\begin{itemize}
	\item Debuggear a nivel de aplicaciones (ring 3)
	\item Más sencillo para principiantes
	\item Posee GUI
\end{itemize}
\begin{center}
\includegraphics[scale=0.25]{OllyDbg2.png}
\end{center}
\end{frame}

\begin{frame}{IDA Pro}

\begin{itemize}
	\item Desemsamblador por excelencia, muchas funciones
	\item Destaca por analizar profundamente
	\item Dificultad media
	\item Posee GUI
\end{itemize}
\begin{center}
\includegraphics[scale=0.2]{incorrect_code.jpg}
\end{center}
\end{frame}

\begin{frame}{Más herramientas}
\begin{itemize}
\item \textbf{Análisis estático de ejecutables} \\ \hspace{4ex}PEview, Dependency Walker, Strings, PEiD...
\vspace{3ex}
\item \textbf{Monitores} \\ \hspace{4ex}Process Explorer, Process Monitor, Wireshark, Regshot...
\vspace{3ex}
\item \textbf{Análisis de malware} \\ \hspace{4ex}Cuckoo Sandbox, AndroGuard, VirusTotal, Koodous...


\end{itemize}

\end{frame}

\section{Ensamblador x86}

\begin{frame}{Un poco más de entorno...}
\begin{center}
\vspace{-3ex}
Simplificadamente, las partes de un programa en memoria:
\vspace{-2ex}
\end{center}
\begin{itemize}
\item \textbf{Código} \\ \hspace{10ex}Instrucciones en código máquina
\vspace{2ex}
\item \textbf{Datos} \\ \hspace{10ex}Variables globales, constantes...
\vspace{2ex}
\item \textbf{Stack} \\ \hspace{10ex}Variables locales, anidamiento de llamadas...
\vspace{2ex}
\item \textbf{Heap} \\ \hspace{10ex}Reserva de memoria bajo demanda

\end{itemize}
\end{frame}

\begin{frame}{Registros principales en x86}
\begin{itemize}
\item \textbf{De propósito general} \\ \hspace{4ex}EAX, EBX, ECX, EDX
\vspace{3ex}
\item \textbf{Stack} \\ \hspace{4ex}ESP y EBP
\vspace{3ex}
\item \textbf{Instruction pointer} \\ \hspace{4ex}EIP
\vspace{3ex}
\item \textbf{Registro de estado} \\ \hspace{4ex}EFLAGS
\end{itemize}
\end{frame}

\begin{frame}{Tipos de instrucciones}
\begin{itemize}
\item \textbf{Aritméticas y transferencia} \\ \hspace{4ex}mov, add, sub, inc, dec, xor...
\vspace{3ex}
\item \textbf{Stack} \\ \hspace{4ex}push, pop, pushad, popad...
\vspace{3ex}
\item \textbf{Comparación y saltos} \\ \hspace{4ex}test, cmp, jmp, je, jne, jl, jge...
\vspace{3ex}
\item \textbf{Llamadas a funciones} \\ \hspace{4ex}call, return...
\end{itemize}
\end{frame}

\begin{frame}{Convenios de llamada: \textit{stdcall}}
\begin{itemize}
\item Se empujan los \textbf{argumentos} de la función en la \textbf{pila} en \\ \hspace{4ex}\textbf{orden inverso}
\vspace{5ex}
\item El responsable de \textbf{limpiar la pila} de los argumentos es la \\ \hspace{4ex}\textbf{función llamada}
\vspace{5ex}
\item Lo retornado por la función (\textbf{return}) se devuelve en el \\ \hspace{4ex}\textbf{registro EAX}
\end{itemize}
\end{frame}

\begin{frame}[fragile]{Bucle for en x86}
\lstset{language={[x86masm]Assembler}}
\begin{lstlisting}

401000 mov ecx,10    #Mueve el valor 10 a ecx
401005 push FF    #Comienzo del cuerpo del bucle
...
# Cuerpo del bucle
...
401039 sub ecx,1    #Resta 1 a ecx
40103C jnz 401005
#Salta a 401005 -> el resultado != cero

\end{lstlisting}
\end{frame}

\begin{frame}{Recursos recomendados}
\begin{itemize}
\item \textbf{Modding y cracking} \\ \hspace{4ex}\textit{The Legend Of Random} \\ \hspace{8ex}\url{http://thelegendofrandom.com/}
\vspace{6ex}
\item \textbf{Análisis de malware} \\ \hspace{4ex}\textit{Practical Malware Analysis} \\ \hspace{8ex}Michael Sikorski y Andrew Honig
\end{itemize}
\end{frame}


\begin{frame}

\begin{center}
\textbf{¡Gracias por vuestra atención!}
\end{center}

\end{frame}

\end{document}
