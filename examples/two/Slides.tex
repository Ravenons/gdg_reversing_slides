\documentclass{beamer}

\usepackage{lmodern}
\usepackage[spanish]{babel}
\usepackage[utf8]{inputenc}
\usepackage[T1]{fontenc}

\setlength{\parskip}{1em}
\usetheme{Warsaw}

\title{Arquitectura Robótica}
\author{Carlos Ledesma Peña y David Zúñiga Noël}
\institute{Programación de Robots}
\date{}

\AtBeginSection[]
{
	\begin{frame}
		\frametitle{Índice}
		\tableofcontents[currentsection]
	\end{frame}
}

\AtBeginSubsection[]
{
	\begin{frame}
		\frametitle{Índice}
		\tableofcontents[currentsection,currentsubsection]
	\end{frame}
}

\begin{document}

\begin{frame}
\titlepage
\end{frame}

\section{Introducción}

\begin{frame}{Puesta en contexto}

Los sistemas robóticos modernos suelen tener una complejidad importante, debido a los siguientes aspectos:

\begin{itemize}
	\item Entorno dinámico
	\item Múltiples sensores y actuadores
	\item Naturaleza asíncrona
	\item Gestión de tareas
	\item Etcétera, etcétera...
\end{itemize}

\end{frame}

\begin{frame}{Concepto de arquitectura robótica}

Para tratar esta complejidad, debemos prestar especial atención a la arquitectura del sistema robótico:

\begin{block}{Arquitectura robótica}
Conjunto de componentes robóticos definidos por su interfaz y función, y los enlaces entre ellos, junto con su topología específica y su naturaleza (coordinación, comunicación, protocolos usados...)
\end{block}

El objetivo de las arquitecturas robóticas es facilitar el desarrollo de sistemas estableciendo ciertas reglas que ayuden a su diseño e implementación, y hacer la programación de los robots lo más sencilla, segura, reutilizable y flexible que sea posible.

\end{frame}

\section{Clasificación de las arquitecturas robóticas}

\begin{frame}{Clasificación de las arquitecturas robóticas}

Las arquitecturas robóticas para robots móviles se clasifican generalmente en los siguientes tipos:

\begin{itemize}
	\item Deliberativas
	\item Reactivas
	\item Híbridas
	\item Basadas en comportamientos
\end{itemize}

Aunque esta clasificación tiene un sentido cronológico importante, no es cierto que las arquitecturas basadas en comportamientos sean posteriores al resto. De hecho, van bastante ligadas a las reactivas, y su evolución es paralela a las arquitecturas híbridas.

\end{frame}

\subsection{Arquitecturas deliberativas}

\begin{frame}{Arquitecturas deliberativas}

\textbf{Piensa, luego actúa}

Basadas en la filosofía Sense-Plan-Act, se elaboraba un plan según lo sentido, que se ejecutaba (secuenciamente). El tiempo de respuesta era elevado, y en entornos dinámicos era un problema.

\begin{tabular}{ l | l }

\emph{Tiempo de respuesta} & Largo \\ \hline
\emph{Modelo del mundo} & Complejo \\ \hline
\emph{Aprendizaje} & Sí \\ \hline
\emph{Ejemplo} & SPA
\end{tabular}

\end{frame}

\subsection{Arquitecturas reactivas}

\begin{frame}{Arquitecturas reactivas}

\textbf{No pienses, actúa}

No hay modelo explícito (''el modelo es el mundo''), lo sentido es procesado de forma sencilla y se envía directamente a los actuadores. Al no tener modelo del mundo, no puede aprender.

\begin{tabular}{ l | l }
\emph{Tiempo de respuesta} & Corto \\ \hline
\emph{Modelo del mundo} & Simple o inexistente \\ \hline
\emph{Aprendizaje} & No \\ \hline
\emph{Ejemplo} & Subsumption
\end{tabular}

\end{frame}

\subsection{Arquitecturas híbridas}

\begin{frame}{Arquitecturas híbridas}

\textbf{Piensa y actúa a la vez (distinta frecuencia y organización)}

Se organiza de abajo arriba en capas según lo abstracto de la tarea. Lo típico son tres capas, una reactiva (bajo nivel), otra deliberativa, alto nivel) y otra que las coordina. El modelo sigue desconectado del mundo real.

\begin{tabular}{ l | l }
\emph{Tiempo de respuesta} & Corto \\ \hline
\emph{Modelo del mundo} & Complejo, en la capa deliberativa \\ \hline
\emph{Aprendizaje} & En la capa deliberativa \\ \hline
\emph{Ejemplo} & 3T
\end{tabular}

\end{frame}

\subsection{Arquitecturas basadas en comportamientos}

\begin{frame}{Arquitecturas basadas en comportamientos}

\textbf{Piensa y actúa a la vez (misma frecuencia y organización)}

Amplía el concepto de la arquitectura reactiva para introducir memoria y alto nivel (comportamientos). Introduce el modelo como comportamientos, cuya interacción permite extender el plan en el tiempo.

\begin{tabular}{ l | l }
\emph{Tiempo de respuesta} & Corto \\ \hline
\emph{Modelo del mundo} & Modelado como comportamientos \\ \hline
\emph{Aprendizaje} &  Introduciendo nuevos comportamientos \\ \hline
\emph{Ejemplo} & Toto
\end{tabular}

\end{frame}

\section{Conclusión}

\begin{frame}{Conclusión}

A día de hoy, y para robots móviles autónomos con una cierta complejidad, las arquitecturas adecuadas son las híbridas y las basadas en comportamientos. Estas dos son igualmente expresivas, si bien las basadas en comportamientos son más fieles a la biología de los procesos cognitivos reales.

Como siempre, la elección particular depende del dominio y la estructura mental del arquitecto.

\end{frame}

\begin{thebibliography}{9}

\begin{frame}{Bibliografía (I)}

\bibitem{mainPaper}
David Kortenkamp, Reid Simmons;
\emph{Robotics System Architectures and Programming}.\\
\url{http://informatica.cv.uma.es/pluginfile.php/128642/mod_folder/content/0/asignados/architectures.pdf}

\bibitem{slidesClass}
Ana Mª Cruz Martín;
\emph{Robots móviles: Arquitecturas robóticas}.\\
\url{http://informatica.cv.uma.es/mod/resource/view.php?id=123519}

\bibitem{slidesBehaviours}
Paul Fitzpatrick;
\emph{Behaviour - Based Control in Mobile Robotics}.\\
\url{http://people.csail.mit.edu/paulfitz/pub/fitzpatrick96behaviour.pdf}

\end{frame}

\begin{frame}{Bibliografía (II)}

\bibitem{slidesOther}
John Kelleher;
\emph{Robot Control Architectures}.\\
\url{http://www.comp.dit.ie/jkelleher/appliedcomputing/week8/RCA.pdf}

\bibitem{slidesBehavioursAdvanced}
Reuven Granot;
\emph{Behavior Based Systems}.\\
\url{http://math.haifa.ac.il/robotics/Presentations/pdf/ch12_bbs.pdf}

\vspace{2em}

\emph{Nota: Las referencias adicionales han sido usadas para una mejor comprensión de la temática.}

\end{frame}

\end{thebibliography}

\end{document}